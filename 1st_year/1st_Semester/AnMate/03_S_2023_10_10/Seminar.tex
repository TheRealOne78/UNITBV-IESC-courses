\documentclass[a4paper, 12pt, notitlepage]{book}
\usepackage{cancel}
\usepackage{amsmath}
\usepackage{amssymb}
\usepackage{spalign}
\usepackage{mathtools}
\usepackage{enumerate}
\begin{document}

    %\title{Seminar 3, Analiz\u{a} Matematic\u{a}}
    %\author{TheRealOne78 --- \textit{BAJCSI Elias-Robert} \and GAROIU Ștefan}
    %\date{10 Octombrie 2023 --- Rev. 1}
    %\maketitle


    \newcommand\revstr{Rev.\@ 1}
    \chapter{Seminar --- 10 Oct.\@ 2023, \revstr}

    \section{Mul\c{t}imea numerelor reale $\mathbb{R}$}
    \subsection{Teorie}
    Fie $A \subseteq \mathbb{R}$ o mul\c{t}ime
    \begin{description}
      \item[Def.] Spunem c\u{a} $A$ este m\u{a}rginit\u{a} dac\u{a}:
            \begin{description}
                    \item $(\exists)\ m,M \in \mathbb{R}$ astfel \^{\i}nc\^{a}t:
                    \item $\underbrace{m}_{\text{Margine inferioar\u{a} (minorant)}} \le a \le \underbrace{M}_{\text{Margine superioar\u{a} (majorant)}}, \forall a \in A$
            \end{description}
      \item[$\inf A$] = cea mai mare margine inferioar\u{a} / cel mai mare minorant
      \item[$\sup A$] = cea mai mic\u{a} margine superioar\u{a} / cel mai mic majorant
      \item[$\min A$] = $\inf A$, dac\u{a} $\inf A \in A$
      \item[$\max A$] = $\sup A$, dac\u{a} $\sup A \in A$
    \end{description}

    \section{Exerci\c{t}ii}
    \subsection{Determina\c{t}i $\inf A$, $\sup A$, $\min A$, $\max A$ pentru:}
    \begin{enumerate}[a.]
      \item $A = ({0,1})$\\
            $\inf A = 0 \in A$\\
            $\sup A = 1 \in A$\\
            $\min A =$ nu exist\u{a}\\
            $\max A =$ nu exist\u{a}

      \item $A = ({-1, 1}] \cup \{2\} $\\
            $\inf A = -1$\\
            $\sup A = 2$\\
            $\min A =$ nu exist\u{a}\\
            $\max A = 2$

      \item $A = ({-\infty, 0}]$\\
            $\inf A = -\infty$\\
            $\sup A = 0$\\
            $\min A =$ nu exist\u{a}\\
            $\max A = 0$

      \item $A = \{x^{2} - 2x +5 \vert x \in \mathbb{R}\}$\\
            $= \text{Im} f= [{y_{v}, \infty})$\\
            $f\ :\ \mathbb{R} \to \mathbb{R}; f(x) = x^{2} - 2x +5$\\
            $y_{v} = \frac{\Delta}{4a}\\
            \Delta = 4-20 = -16 \Rightarrow y_{v} = \frac{16}{4} = 4$\\

            $\text{Im} f= [{y_{v}, \infty}) = [{4, \infty})$\\

            $\inf A = \infty$\\
            $\sup A = 4$\\
            $\min A = 4$ \\
            $\max A = $ nu exist\u{a}

      \item $A = \{x^{2} + 2x - 1 \vert x \in \mathbb{R}\}$\\
            $\text{Im} f= [{-\infty, \frac{-\Delta}{4a}})$\\
            $\Delta = 4-4 = 0 \Rightarrow \text{Im} f= ({-\infty, 0}]$

            $\inf A = -\infty$\\
            $\sup A = 0$\\
            $\min A =$ nu exist\u{a}\\
            $\max A = 0$

      \item $A = \{x \in \mathbb{R} \vert 4x^{2} +3x-1 > 0\}$\\
            $f\ :\ \mathbb{R} \to \mathbb{R}, f(x) = 4x^{2} +3x-1$\\
            $f(x) = 0$\\
            $\Delta = 9+16 = 25$\\
            $x_{1,2} = \frac{-1 \pm 5}{8}$\\
            $x_{1} = -1;\ x_{2} = \frac{1}{4} \Rightarrow $\\

            $\inf A = -\infty$\\
            $\sup A = \infty$\\

            \begin{table}[h]
              %\centering
              \begin{tabular}{l|llllllll}
                $x$    & $-\infty$ &   & -1 &   &   & $\frac{1}{4}$ &   & $\infty$  \\
                \hline
                $f(x)$ & +       & + & 0  & - & - & 0           & + & +
              \end{tabular}
            \end{table}

            $A = (-\infty, -1) \cup \left( \frac{1}{4}, \infty \right)$\\
            $\inf A = -\infty$\\
            $\sup A = \infty$
    \end{enumerate}

    \subsection{Care din urm\u{a}toarele mul\c{t}imi este m\u{a}rginit\u{a}?}
    \begin{enumerate}[\quad a.]
      \item $A = \{ \frac{1}{x} \vert x \in (0,\infty) \}$\\[5pt]
            $0 < x < y \Leftrightarrow 0 < \frac{1}{x} < \infty$\\[5pt]
            $\underbrace{\frac{1}{0}}_{\infty} > \frac{1}{x} > \frac{1}{\infty} $\\[5pt]
            \begin{tabular}{ll}
              $A = \{ \frac{1}{x} \vert x \in (0,\infty) \} = (0, \infty) \Rightarrow$ & $\inf A = 0$ --- m\u{a}rginit\\
                                                                                       & $\sup A = \infty$ --- nu este m\u{a}rginit
            \end{tabular}
      \item $A = \{ \frac{n+1}{n+2} \arrowvert n \in \mathbb{N} \}$\\[5pt]
            $0 < \frac{n+1}{n+2} < 1 \Rightarrow A$ --- m\u{a}rginit\u{a}
      \item $A = \{ \sin n | n \in \mathbb{N} \} = [-1, 1) \Rightarrow A$ --- m\u{a}rginit\u{a}
      \item $A = \{ \frac{n^{2}}{n+1} | n \in \mathbb{N} \}$\\[5pt]
            $\frac{n^{2}}{n+1} > 0$\\[5pt]
            $\frac{n^{2}}{n+1} < \frac{n^{2}}{n} = n$\\[5pt]
            $n \in \mathbb{N},\quad \mathbb{N}$ --- nem\u{a}rginit\u{a} $\; \Rightarrow \; A$ --- nu este m\u{a}rginit\u{a}
    \end{enumerate}

    \subsection{$A, B \subset \mathbb{R},\;\; A, B \ne \emptyset,\;\; A, B$ --- m\u{a}rginite\\
      Ar\u{a}ta\c{t}i c\u{a}:}
    $\min \{ \inf A, \inf B \} = \text{Im}(A \cup B) \le \sup (A \cup B) = \max \{ \sup A, \sup B \}$\\[5pt]
    Presupunem \;\; $\text{Im} A \le \inf B \Rightarrow \min \{ \inf A, \inf B \} = \inf A$\\[5pt]
    \begin{math}
      \begin{rcases}
        \inf A \le a,\; \forall a \in A\\
        \inf B \le b,\; \forall b \in B\\
        \inf A \le B
      \end{rcases}
      \Rightarrow \inf A \le b,\; \forall\ b \in B\\[5pt]
    \end{math}
    $\inf (A \cup B) = \inf A$\\[5pt]

    \begin{center}
      Def.: Fie $x_{0} \in \mathbb{R}$. Spunem c\u{a} mul\c{t}imea $V$ este o vecin\u{a}tate a lui $x_{0}$ dac\u{a}:
      \[ \left(x_{0} - \varepsilon, x_{o} + \varepsilon \right) \subset V \]
      $V_{x_{0}} = $ mul\c{t}imea tuturor vecin\u{a}t\u{a}\c{t}ilor lui $x_{0}$.\\
    \end{center}

    \subsection{Preciza\c{t}i care din urm\u{a}toarele mul\c{t}imi sunt vecin\u{a}t\u{a}\c{t}i pentru $x_{0}$}
    \begin{enumerate}[a.]
      \item $(-1, \infty) \in V_{o}$\\[5pt]
            $(-\varepsilon, \varepsilon) \subset (-1, \infty)$\\[5pt]
            $\varepsilon = 1 \; \Rightarrow \; (-1, 1) \subset (-1, \infty)$
      \item $(10, 11) \in V_{10}$
            $(\underbrace{10 - \varepsilon}_{< 10}, 10 + \varepsilon) \not\subset (10,11)$
      \item $\{ 0,1,2 \} \in V_{1 } \quad $ (NU --- o mul\c{t}ime de elemente nu are vecin\u{a}t\u{a}\c{t}i)
      \item $\overline{\mathbb{R}} \in V_{-\infty}$ \quad (DA)\\[5pt]
            $(-\infty, \infty) < \underbrace{\left[-\infty, \infty \right]}_{\overline{\mathbb{R}}}$
      \item $\mathbb{Q} \in V_{0}$ \quad (NU)
    \end{enumerate}
    \section{\c{S}iruri de numere reale}
    Def.: Un \c{s}ir de numere reale este o func\c{t}ie $f : \mathbb{N} \to M,\; M \subseteq \mathbb{R}$\\[5pt]
    $\mathbb{N} \ni n \to f(n) = \underbrace{x_{n}}_{\text{termenul general al \c{s}irului}} \in M$
    \subsection{Exemple}
    \begin{enumerate}
      \item $x_n = n, \; \forall\ n \in \mathbb{N}$
      \item $x_{n} = 2 \cdot n,\; \forall\ n \in \mathbb{N}$ (sub\c{s}ir al \c{s}irului --- nr.\ naturale)
      \item $x_{n} = 2n+1,\; \forall\ n \in \mathbb{N}$ (sub\c{s}ir al \c{s}irului --- nr.\ naturale)
    \end{enumerate}

    \subsection{Nota\c{t}ie}
    \[ (x_{n})_{n} \in \mathbb{N} \text{ --- \c{s}ir de numere naturale} \]

    \subsection{Defini\c{t}ie}
    Spunem c\u{a} \c{s}irul ${(x_{n})}_{n} \in \mathbb{N}$ este monoton dac\u{a}:
    \begin{enumerate}
      \item $x_n \text{ --- cresc\u{a}tor, } \forall n \in \mathbb{N}:$\\[5pt]
            $x_{0} \le x_{1} \le x_{2} \le \ldots \le x_{n} \le x_{n+1} \le \ldots \; \forall\ n \in \mathbb{N}$\\

            sau\\
      \item $x_{n} \text{ descresc\u{a}tor}, \forall\ n \in \mathbb{N}$\\[5pt]
            $x_{0} \ge x_{1} \ge x_{2} \le \ldots \ge x_{n} \ge x_{n+1} \ge \ldots \; \forall\ n \in \mathbb{N}$\\
    \end{enumerate}

    \subsection{Criteriu}
    $(x_{n})\ n \in \mathbb{N} \text{ este:}$
    \begin{enumerate}[\quad a.]
      \item Cresc\u{a}tor dac\u{a}:\\[5pt]
            $x_{n+1} - x_{n} \ge 0, \forall\ n \in \mathbb{N}$
      \item Descresc\u{a}tor dac\u{a}:\\[5pt]
            $x_{n+1} - x_{n} \le 0, \forall\ n \in \mathbb{N}$
    \end{enumerate}

    \subsection{Exerci\c{t}ii}
    \begin{enumerate}
      \item Studia\c{t}i monotonia \c{s}irurilor
            \begin{enumerate}[\quad a.]
              \item $x_{n} = \frac{2n+1}{4n+3}, n \in \mathbb{N}$\\[5pt]
                    $x_{n+1} = \frac{2(n+1) + 1}{4(n+1) +3} = \frac{2n+3}{4n+7}$\\[5pt]
                    \begin{tabular}{ll}
                      $x_{n+1} - x_{n}$ & $= \frac{2n+3}{4n+7} - \frac{2n+1}{4n+3} =$\\[5pt]
                                     & $= \frac{(2n+3)(4n+3)-(2n+1)(4n+7)}{(4n+3)(4n+7)}=$\\[5pt]
                                     & $= \frac{\cancel{8n^{2}} + \cancel{18n} + 9 - \cancel{8n^{2}} - \cancel{18n} - 7}{(4n+3)(4n+7)}=$\\[5pt]
                                     & $= \frac{2}{(4n+3)(4n+7)} > 0$
                    \end{tabular}\\[5pt]
                    \begin{math}
                      \begin{rcases}
                        n \in \mathbb{N}\\[5pt]
                        2 (???)\\[5pt]
                      \end{rcases}
                      \Rightarrow x_{n+1} - x_{n} > 0, \;\; x_{n} \text{ monoton cresc\u{a}tor}
                    \end{math}

              \item $x_{n} \frac{1}{n+1} + \frac{1}{n+2} + \frac{1}{n+3} \ldots \frac{1}{2n},\;\; n \ge 1$\\[5pt]
                    $x_{n+1} = \frac{1}{n+2} + \frac{1}{n+3} \ldots \frac{1}{2n} + \frac{1}{2n+1} + \frac{1}{2n+2}$\\[5pt]
                    \begin{tabular}{ll}
                    $x_{n+1} - x_{n}$ & $= \frac{1}{n+2} + \frac{1}{n+3} \ldots \frac{1}{2n} + \frac{1}{2n+1} + \frac{1}{2n+2} - \frac{1}{n+1} - \frac{1}{n+2} \ldots - \frac{1}{2n}=$\\[5pt]
                                      & $= \frac{1}{2n+1} + \frac{1}{2n+2} - \frac{1}{n+1} =$\\[5pt]
                                      & $= \frac{1}{2n+1} - \frac{1}{2n+2} > 0,\;\; x_{n} \text{ monoton cresc\u{a}tor}$
                    \end{tabular}

              \item $x_{n} = (-1)^{n} + \frac{1}{n}, \; n \ge 1$\\[5pt]
                    $n = 1 \Rightarrow x_{1} = -1 +1 = 0$\\[5pt]
                    $n = 2 \Rightarrow x_{2} = -1 + {1 \over 2} = \frac{3}{2} $\\[5pt]
                    $n = 3 \Rightarrow x_{3} = -1 + {1 \over 3} = - \frac{2}{3}$\\[5pt]
                    \begin{math}
                      \begin{rcases}
                        x_{1} < x_{2}\\[5pt]
                        x_{2} > x_{3}
                      \end{rcases}
                      \Rightarrow x_{n} \text{ --- nu este monoton }
                    \end{math}

              \item $x_{n} = \cos (\pi \cdot n), \forall\ n \in \mathbb{N}$\\[5pt]
                    $n = 0 \Rightarrow x_{0} = \cos 0    = 1$\\[5pt]
                    $n = 1 \Rightarrow x_{1} = \cos \pi  = -1 $\\[5pt]
                    $n = 2 \Rightarrow x_{2} = \cos 2\pi = 1 $\\[5pt]
                    $x_{n} = {(-1)}^{2}$ --- nu este monoton

              \item $x_{n} = \frac{1 \cdot 4}{3^{2}} \cdot \frac{2 \cdot 5}{4^{2}} \ldots \frac{n(n+3)}{{(n+2)}^{2}}$\\[5pt]
                      $x_{n+1}= \frac{1 \cdot 4}{3^{2}} \cdot \frac{2 \cdot 5}{4^{2}} \ldots \frac{n(n+3)}{{(n+2)}^{2}} \cdot \frac{(n+1)(n+4)}{{(n+3)}^{2}}$\\[5pt]

                    \begin{tabular}{ll}
                      $\frac{x_{n+1}}{x_{n}}$ & $= \frac{1 \cdot 4 \cdot 2 \cdot 5 \ldots n(n+3)(n+1)(n+4)}{3^{2} \cdot 4^{2} \ldots {(n+2)}^{2} \cdot {(n+3)}^{2}} \cdot \frac{3^{2} \cdot 4^{2} \ldots {(n+2)}^{2}}{1 \cdot 4 \cdot 2 \cdot 5 \ldots n(n+3)} =$\\[5pt]
                                              & $= \frac{(n+1)(n+4)}{(n+3)^{2}}=$\\[5pt]
                                              & $= \frac{n^{2} + 5n + 4}{n^{2} + 6n + 9} < 1$
                    \end{tabular}\\[5pt]
                    $\Rightarrow x_{n+1} < x_{n} \; \Rightarrow \; (x_{n})_{n} \in \mathbb{N} $ --- monoton descresc\u{a}tor
            \end{enumerate}
    \end{enumerate}

\end{document}
