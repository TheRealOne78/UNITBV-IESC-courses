% Created 2026-02-16 Mon 20:02
% Intended LaTeX compiler: lualatex
\documentclass[11pt]{article}
\usepackage{amsmath}
\usepackage{fontspec}
\usepackage{graphicx}
\usepackage{longtable}
\usepackage{wrapfig}
\usepackage{rotating}
\usepackage[normalem]{ulem}
\usepackage{capt-of}
\usepackage{hyperref}
\usepackage{minted}
\usepackage{fontspec}
\setmainfont{UT Sans}[
BoldFont={UT Sans Bold},
UprightFont={UT Sans Regular},
FontFace={m}{n}{UT Sans Medium}
]
% Fallback fake italic using a slant
\usepackage{etoolbox}
\newcommand{\fakeitalic}[1]{{\addfontfeatures{FakeSlant=0.18}#1}}
\AtBeginDocument{\renewcommand{\emph}[1]{\fakeitalic{#1}}}
\usepackage{minted}
\usepackage[a4paper,margin=3cm]{geometry}
\usepackage{lastpage}
\author{BAJCSI Elias-Robert}
\date{2026-02-16 Mon}
\title{N11 - teme lab01}
\hypersetup{
 pdfauthor={BAJCSI Elias-Robert},
 pdftitle={N11 - teme lab01},
 pdfkeywords={},
 pdfsubject={},
 pdfcreator={Emacs 30.2 (Org mode 9.7.39)}, 
 pdflang={English}}
\begin{document}

\maketitle
\section{Listati continutul directoarelor /, /bin, /usr, /etc, /usr/include. Acolo unde este cazul, paginati listarea (ls | less). cautati, in fisierul /usr/include/stdio.h, textul printf.}
\label{sec:org0c0daf3}
\subsection{List}
\label{sec:orga96baad}
\begin{minted}[]{sh}
ls /
ls /bin | less
ls /usr
ls /etc | less
ls /usr/include | less
\end{minted}
\subsection{Concat}
\label{sec:org7161d12}
\begin{minted}[]{sh}
cat /usr/include/stdio.h | grep printf
\end{minted}
\section{Creati, in directorul personal, urmatoarea structura de directoare si fisiere:}
\label{sec:orga117ae1}
\begin{verbatim}
(dir. personal)
|
+-- abc
| +-- x (fisier)
| +-- y (fisier)
| +-- t1 (fisier)
| +-- t2 (fisier)
| +-- t3 (fisier)
| +-- t (director)
| +-- a (fisier)
| +-- b (fisier)
|
+-- zz (director)
| +-- x (fisier)
|
+-- tt (director)
\end{verbatim}

\begin{minted}[]{sh}
mkdir -p abc/t zz/x tt

cd ./abc; touch x y t{1..3} a b; cd ..
\end{minted}
\section{Copiati directorul abc cu tot continutul sau (recursiv) ca subdirector al lui zz (va rezulta un subdirector abc in zz)}
\label{sec:org1bc113a}
\begin{minted}[]{sh}
cp -r ./abc ./zz
\end{minted}
\section{Copiati continutul lui abc in directorul zz fara a suprascrie fisierele cu acelasi nume (x, in cazul nostru)}
\label{sec:orge7d2e64}
\begin{minted}[]{sh}
cp --update=none -r ./abc/* ./zz
\end{minted}
\section{copiati fisierele t1 si t2 din abc in tt (folosind specificator generic)}
\label{sec:org0ea5d35}
\begin{minted}[]{sh}
cp ./abc/t{1..2} tt
\end{minted}
\section{Creati un director pe care sa va dati dreptul x fara a avea dreptul r. Creati in el un fisier. Ce observati? Dati-va apoi dreptul r si luati-va dreptul x. Ce observati ?}
\label{sec:org8cbfba6}
\subsection{Execute, no read}
\label{sec:org73be716}
\begin{minted}[]{sh}
mkdir -m 300 foo
touch foo/bar
\end{minted}
Pot crea fișiere în directorul respectiv (deoarece în acest caz, mi-am dat și permisiunea de scriere [write]), însă nu pot lista fișierele.
\subsection{Read, no execute}
\label{sec:org58afc33}
\begin{minted}[]{sh}
chmod 600 foo
\end{minted}
Acum nu mai am cum să schimb directorul în ``\$(pwd)/foo'', și nu am cum sa creez conținut în el, deoarece drepturile de execuție în directorul respectiv îmi sunt blocate.
\section{Dati drepturile potrivite astfel incat oricine sa poata vizualiza continutul directoarelor abc si abc/t, sa poata adauga fisiere in abc/t, sa poata citi fisierele x, y, t1, t2, t3 din abc dar sa nu poata citi fisierele a si b din abc/t}
\label{sec:orgd25e0bd}
\begin{minted}[]{sh}
chmod a+rx abc abc/t
chmod a+w abc/t

for f in x y t{1..3}; do
    chmod a+r "abc/${f}"
done

for f in a b; do
    chmod a-r "abc/t/${f}"
done
\end{minted}
\section{Listati in format lung fisierele t, t1, t2, t3 din abc (sa se vada drepturile de acces asupra lui t, nu asupra fisierelor din el}
\label{sec:org432ba6b}
\begin{minted}[]{sh}
for f in t t1 t2 t3; do
    ls -l abc
done
\end{minted}
\section{Comanda copy /dev/zero /dev/null este un fel de 'ciclu infinit' (nu se termina). Lansati-o, mutati-o in background, listati procesele active, terminati comanda (in ambele variante: mutata in foreground si oprita cu \^{}C, sau cu kill).}
\label{sec:org74779bb}

\begin{minted}[]{sh}
cp /dev/zero /dev/null &
ps -aux | grep "cp /dev/zero /dev/null"

# First variant
fg
#C c

# Re-run
cp /dev/zero /dev/null &
kill $! # Kill last process ID
\end{minted}
\section{Arhivati structura de directoare folosind utilitarul zip sau utilitarele tar si gz.}
\label{sec:org8d14a49}
\subsection{Zip}
\label{sec:org3fc8aa5}
\begin{minted}[]{sh}
zip -r structure.zip src
\end{minted}
\subsection{GNU Tar + GNU Zip}
\label{sec:org4c9fd7a}

\begin{minted}[]{sh}
tar -cpzf structure.tar.gz src

# Or pipe
tar -cp src | gzip > structure2.tar.gz
\end{minted}
\section{Stergeti directoarele si refaceti structura pornind de la arhiva. Verificati conservarea datei creerii si drepturilor de acces.}
\label{sec:orgcad485c}
\subsection{Remove dirs}
\label{sec:org98303b4}
\begin{minted}[]{sh}
rm -rf src
\end{minted}
\subsection{Unarchive}
\label{sec:orgdc6068d}
\subsubsection{ZIP}
\label{sec:org2285437}
\begin{minted}[]{sh}
unzip structure.zip
\end{minted}
\subsubsection{GNU Tar + GNU Zip}
\label{sec:orgdb1144d}
\begin{minted}[]{sh}
tar -xzf structure.tar.gz

# Or pipe
gunzip -c structure.tar.gz | tar -xf -
\end{minted}
\end{document}
